\begin{Exercise}[label={0001}, difficulty={1}, origin={bash}]
  Dentro de sua pasta pessoal, crie um arquivo chamado \verb+ex0001.lpoo+
  contendo a listagem detalhado do diretório \verb+/usr/bin+, usando apenas um
  comando. 
\end{Exercise}
\begin{Exercise}[label={0002}, difficulty={2}, origin={bash}]
  Usando apenas um comando, liste todos os arquivos da pasta raiz, separe apenas
  os que tenham a letra \verb+b+, e substitua a palavra \verb+in+ por
  \verb+omba+.
\end{Exercise}
\begin{Exercise}[label={0003}, difficulty={1}, origin={gawk}]
  Considerando o arquivo \lstinline+example1.apps+ cujas linhas são
  apresentadas abaixo:
  \lstinputlisting{aux/example1.apps}
  \begin{enumerate}
    \item Escreva uma chamada do \lstinline+gawk+ que imprima a linha em que
      \lstinline+Upper+ é informado.
    \item Escreva uma chamada do \lstinline+gawk+ que imprima a linha em que
      \lstinline+Lower+ é informado.
    \item Escreva uma chamada do \lstinline+gawk+ que imprima apenas o inteiro
      correspondente ao nível do \lstinline+Debug+.
  \end{enumerate}
\end{Exercise}
\begin{Answer}[ref={0003}]
  \begin{enumerate}
    \item \lstinline+awk '/Upper/ {print $0}' example1.apps+
    \item \lstinline+awk '/Lower/ {print $0}' example1.apps+
    \item \lstinline+awk '/Debug/ {print $3}' example1.apps+
  \end{enumerate}
\end{Answer}
\begin{Exercise}[label={0004}, difficulty={1}, origin={gawk}]
  Considerando o arquivo \lstinline+example3.apps+ cujas linhas são
  apresentadas abaixo:
  \lstinputlisting{aux/example3.apps}
  \begin{enumerate}
    \item Escreva uma chamada do \lstinline+gawk+ e outra do \lstinline+sed+ que
      mude o nível de \lstinline+Debug+ de 3 para 4.
    \item Escreva uma chamada do \lstinline+gawk+ e outra do \lstinline+sed+ que
      mude a tolerância do passo de $1e-5$ para $1e-8$.
  \end{enumerate}
\end{Exercise}
\begin{Answer}[ref={0004}]
  \begin{enumerate}
    \item \lstinline+awk '/Debug/ {print $0}' example3.apps | sed 's/3/4/'+
    \item \lstinline+awk '/Step Tolerance/ {print $0}' example3.apps | sed 's/5/8/'+
  \end{enumerate}
\end{Answer}
\begin{Exercise}[label={0005}, difficulty={1}, origin={gawk}]
  Considerando o arquivo \lstinline+example3.apps+ cujas linhas são
  apresentadas abaixo:
  \lstinputlisting{aux/pspdoc.log}
  \begin{enumerate}
    \item Escreva uma chamada do \lstinline+gawk+ que informe o nome do
      problema apenas uma vez.
    \item Escreva uma chamada do \lstinline+gawk+ que informe o número de
      variáveis.
    \item Escreva uma chamada do \lstinline+gawk+ para determinar se o algoritmo
      convergiu.
    \item Escreva uma chamada do \lstinline+gawk+ que informe o número de
      iterações.
  \end{enumerate}
\end{Exercise}
\begin{Answer}[ref={0005}]
  \begin{enumerate}
    \item \lstinline+awk '/Problem name/ {name = $3}; END {print name}' pspdoc.log+
    \item \lstinline+awk '/Number of Variables/ {print $4}' pspdoc.log+
    \item \lstinline+awk '/The Algorithm has Converged/ {print "Convergiu"}' pspdoc.log+
    \item \lstinline+awk '/Number of Iterations/ {print $5}' pspdoc.log+
  \end{enumerate}
\end{Answer}
\begin{Exercise}[label={0006}, difficulty={2}, origin={gawk}]
  Considerando o arquivo \lstinline+rhomax-table.tex+ cujas linhas são
  apresentadas abaixo:
  \lstinputlisting{aux/pspdoc.log}
  Escreva um programa utilizando o \lstinline+gawk+ que informa o número de
  problemas em que o campo \lstinline+res+ é \lstinline+con+ e \lstinline+unc+.
\end{Exercise}
\begin{Answer}[ref={0006}]
  \begin{lstlisting}
BEGIN {c = 0; u = 0}
/con/ {c = c + 1}
/unc/{u = u + 1}
END {printf "con = %d\nunc = %d", c, u}
  \end{lstlisting}
\end{Answer}
\begin{Exercise}[label={0007}, difficulty={2}, origin={gawk}]
  Considerando o arquivo \lstinline+dcicpp.col+ cujas linhas são
  apresentadas abaixo:
  \lstinputlisting{aux/dcicpp.col}
  Escreva um programa utilizando o \lstinline+gawk+ que informa o número de
  problemas que convergiram e a soma do tempo de execução dos mesmos.
\end{Exercise}
\begin{Answer}[ref={007}]
  \begin{lstlisting}
BEGIN{c = 0; t = 0}
/Converged/ {c = c + 1; t = t + $3}
END{printf "Converged: %d\nTotal Time: %f\n",  c, t}
  \end{lstlisting}
\end{Answer}
