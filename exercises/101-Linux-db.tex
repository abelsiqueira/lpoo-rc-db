\begin{Exercise}[label={0001},difficulty={0},origin={bash}]
  Dentro de sua pasta pessoal, crie um arquivo chamado \lstinline+ex0001.lpoo+
  contendo a listagem detalhado do diretório \lstinline+/usr/bin+, usando apenas um
  comando. 
\end{Exercise}
\begin{Answer}[ref={0001}]
\begin{lstlisting}
ls -l /usr/bin > ~/ex0001.lpoo
\end{lstlisting}
\end{Answer}

\begin{Exercise}[label={0002},difficulty={2},origin={bash}]
  Usando apenas um comando, liste todos os arquivos da pasta raiz, separe apenas
  os que tenham a letra \lstinline+b+, e substitua a palavra \lstinline+in+ por
  \lstinline+omba+.
\end{Exercise}
\begin{Answer}[ref={0002}]
\begin{lstlisting}
ls / | grep b | sed 's/in/omba/g'
\end{lstlisting}
\end{Answer}

\begin{Exercise}[label={0003}, difficulty={1}, origin={gawk}]
  Considerando o arquivo \lstinline+example1.apps+ cujas linhas são
  apresentadas abaixo:
  \lstinputlisting{aux/example1.apps}
    \Question Escreva uma chamada do \lstinline+gawk+ que imprima a linha em que
      \lstinline+Upper+ é informado.
    \Question Escreva uma chamada do \lstinline+gawk+ que imprima a linha em que
      \lstinline+Lower+ é informado.
    \Question Escreva uma chamada do \lstinline+gawk+ que imprima apenas o inteiro
      correspondente ao nível do \lstinline+Debug+.
\end{Exercise}
\begin{Answer}[ref={0003}]
  \Question \lstinline+awk '/Upper/ {print $0}' example1.apps+
  \Question \lstinline+awk '/Lower/ {print $0}' example1.apps+
  \Question \lstinline+awk '/Debug/ {print $3}' example1.apps+
\end{Answer}

\begin{Exercise}[label={0004}, difficulty={1}, origin={gawk}]
  Considerando o arquivo \lstinline+example3.apps+ cujas linhas são
  apresentadas abaixo:
  \lstinputlisting{aux/example3.apps}
    \Question Escreva uma chamada do \lstinline+gawk+ e outra do \lstinline+sed+ que
      mude o nível de \lstinline+Debug+ de 3 para 4.
    \Question Escreva uma chamada do \lstinline+gawk+ e outra do \lstinline+sed+ que
      mude a tolerância do passo de $1e-5$ para $1e-8$.
\end{Exercise}
\begin{Answer}[ref={0004}]
    \Question \lstinline+awk '/Debug/ {print $0}' example3.apps | sed 's/3/4/'+
    \Question \lstinline+awk '/Step Tolerance/ {print $0}' example3.apps | sed 's/5/8/'+
\end{Answer}

\begin{Exercise}[label={0005}, difficulty={1}, origin={gawk}]
  Considerando o arquivo \lstinline+example3.apps+ cujas linhas são
  apresentadas abaixo:
  \lstinputlisting{aux/pspdoc.log}
    \Question Escreva uma chamada do \lstinline+gawk+ que informe o nome do
      problema apenas uma vez.
    \Question Escreva uma chamada do \lstinline+gawk+ que informe o número de
      variáveis.
    \Question Escreva uma chamada do \lstinline+gawk+ para determinar se o algoritmo
      convergiu.
    \Question Escreva uma chamada do \lstinline+gawk+ que informe o número de
      iterações.
\end{Exercise}
\begin{Answer}[ref={0005}]
    \Question \lstinline+awk '/Problem name/ {name = $3}; END {print name}' pspdoc.log+
    \Question \lstinline+awk '/Number of Variables/ {print $4}' pspdoc.log+
    \Question \lstinline+awk '/The Algorithm has Converged/ {print "Convergiu"}' pspdoc.log+
    \Question \lstinline+awk '/Number of Iterations/ {print $5}' pspdoc.log+
\end{Answer}

\begin{Exercise}[label={0006}, difficulty={2}, origin={gawk}]
  Considerando o arquivo \lstinline+rhomax-table.tex+ cujas linhas são
  apresentadas abaixo:
  \lstinputlisting{aux/rhomax-table.tex}
  Escreva um programa utilizando o \lstinline+gawk+ que informa o número de
  problemas em que o campo \lstinline+res+ é \lstinline+con+ e \lstinline+unc+.
\end{Exercise}
\begin{Answer}[ref={0006}]
  \begin{lstlisting}
BEGIN {c = 0; u = 0}
/con/ {c = c + 1}
/unc/{u = u + 1}
END {printf "con = %d\nunc = %d", c, u}
  \end{lstlisting}
\end{Answer}

\begin{Exercise}[label={0007}, difficulty={2}, origin={gawk}]
  Considerando o arquivo \lstinline+dcicpp.col+ cujas linhas são
  apresentadas abaixo:
  \lstinputlisting{aux/dcicpp.col}
  Escreva um programa utilizando o \lstinline+gawk+ que informa o número de
  problemas que convergiram e a soma do tempo de execução dos mesmos.
\end{Exercise}
\begin{Answer}[ref={007}]
  \begin{lstlisting}
BEGIN{c = 0; t = 0}
/Converged/ {c = c + 1; t = t + $3}
END{printf "Converged: %d\nTotal Time: %f\n",  c, t}
  \end{lstlisting}
\end{Answer}

\begin{Exercise}[label={0008}, difficulty={1}, origin={git}]
  Escreva a chamada do Git que inicializar o repositório.
\end{Exercise}
\begin{Answer}[ref={0008}]
  \begin{lstlisting}
$ git init
  \end{lstlisting}
\end{Answer}

\begin{Exercise}[label={0009}, difficulty={1}, origin={git}]
  Considere a situação descrita abaixo:
  \begin{lstlisting}
$ git status
# On branch master
#
# Initial commit
#
# Untracked files:
#   (use "git add <file>..." to include in what will be committed)
#
#       hello.c
nothing added to commit but untracked files present (use "git add" to track)
  \end{lstlisting}
  Diga qual arquivo não foi adicionado ao repositório e como fazê-lo.
\end{Exercise}
\begin{Answer}[ref={0009}]
  O arquivo \lstinline+hello.c+ não foi adicionado ao repositório, para fazê-lo
  utiliza-se:
  \begin{lstlisting}
$ git add hello.c
  \end{lstlisting}
\end{Answer}

\begin{Exercise}[label={0010}, difficulty={1}, origin={git}]
  Considere a situação descrita abaixo:
  \begin{lstlisting}
$ git status
# On branch master
#
# Initial commit
#
# Changes to be committed:
#   (use "git rm --cached <file>..." to unstage)
#
#       new file:   hello.c
#
  \end{lstlisting}
  Diga se é possível criar um commit e como fazê-lo.
\end{Exercise}
\begin{Answer}[ref={0010}]
  Sim, é possível e para fazê-lo utiliza-se:
  \begin{lstlisting}
$ git commit -m "Adicionado arquivo hello.c"
  \end{lstlisting}
\end{Answer}

\begin{Exercise}[label={0011}, difficulty={1}, origin={git}]
  Considere a situação descrita abaixo:
  \begin{lstlisting}
$ git status
# On branch master
#
# Initial commit
#
# Changes to be committed:
#   (use "git rm --cached <file>..." to unstage)
#
#       new file:   hello.c
#
  \end{lstlisting}
  Diga se é possível criar um commit e como fazê-lo.
\end{Exercise}
\begin{Answer}[ref={0011}]
  Sim, é possível e para fazê-lo utiliza-se:
  \begin{lstlisting}
$ git commit -m "Adicionado arquivo hello.c"
  \end{lstlisting}
\end{Answer}

\begin{Exercise}[label={0012}, difficulty={1}, origin={git}]
  Considere a situação descrita abaixo:
  \begin{lstlisting}
$ git status
# On branch master
# Changes not staged for commit:
#   (use "git add <file>..." to update what will be committed)
#   (use "git checkout -- <file>..." to discard changes in working directory)
#
#       modified:   hello.c
#
no changes added to commit (use "git add" and/or "git commit -a")
  \end{lstlisting}
  Diga como fazer para descobrir o que mudou no arquivo \lstinline+hello.c+.
\end{Exercise}
\begin{Answer}[ref={0012}]
  \begin{lstlisting}
$ git diff -- hello.c
  \end{lstlisting}
\end{Answer}

\begin{Exercise}[label={0013}, difficulty={1}, origin={git}]
  Diga qual o comando que pode ter gerado a saída abaixo:
  \begin{lstlisting}
* 2228f1f Troca de "mundo" por "marte"
* 81be5db Adicionado arquivo hello.c
  \end{lstlisting}
\end{Exercise}
\begin{Answer}[ref={0013}]
  \begin{lstlisting}
$ git log --graph --oneline
  \end{lstlisting}
\end{Answer}

\begin{Exercise}[label={0014}, difficulty={1}, origin={bash}]
  O programa \lstinline+cat+ copia o conteúdo dos arquivos passados como
  argumentos para a saída padrão.

  \Question Escreva o comando para que escreva na tela o arquivo
  \lstinline+/etc/hostname+.
  \Question Escreva o comando para que escreva na tela o arquivo
  \lstinline+/etc/hostname+ e depois o arquivo \lstinline+/etc/hosts+.
\end{Exercise}
\begin{Answer}[ref={0014}]
  \Question \lstinline+cat /etc/hostname+
  \Question \lstinline+cat /etc/hostname /etc/hosts+
\end{Answer}

\begin{Exercise}[label={0015}, difficulty={1}, origin={bash}]
  O programa \lstinline+head+ copia o início do conteúdo dos arquivos passados
  como argumentos para a saída padrão.

  O programa \lstinline+tail+ copia o final do conteúdo dos arquivos passados
  como argumentos para a saída padrão.

  \Question Escreva o comando para que escreva na tela o início do arquivo
  \lstinline+/etc/passwd+.
  \Question Escreva o comando para que escreva na tela o final do arquivo
  \lstinline+/etc/passwd+.
\end{Exercise}
\begin{Answer}[ref={0015}]
  \Question \lstinline+head /etc/passwd+
  \Question \lstinline+tail /etc/passwd+
\end{Answer}

\begin{Exercise}[label={0016}, difficulty={1}, origin={bash}]
  O programa \lstinline+wc+ recupera várias informações de um arquivo de texto.

  \Question Escreva o comando que imprime na saída padrão o número de
  caracteres do arquivo \lstinline+/etc/hosts+.
  \Question Escreva o comando que imprime na saída padrão o número de palavras
  do arquivo \lstinline+/etc/hosts+.
  \Question Escreva o comando que imprime na saída padrão o número de linhas
  do arquivo \lstinline+/etc/hosts+.
\end{Exercise}
\begin{Answer}[ref={0016}]
  \Question \lstinline+wc -m /etc/hosts+
  \Question \lstinline+wc -w /etc/hosts+
  \Question \lstinline+wc -l /etc/hosts+
\end{Answer}

\begin{Exercise}[label={0017}, difficulty={1}, origin={bash}]
  O programa \lstinline+ls+ lista os arquivos e pastas de um diretório.

  \Question Escreva o comando que imprime na saída a lista do conteúdo do
  diretório atual.
  \Question Escreva o comando que imprime na saída a lista do conteúdo da
  sua home.
  \Question Escreva o comando que imprime na saída a lista do conteúdo da
  raiz do sistema.
\end{Exercise}
\begin{Answer}[ref={0017}]
  \Question \lstinline+ls+
  \Question \lstinline+ls ~+
  \Question \lstinline+ls /+
\end{Answer}

\begin{Exercise}[label={0018}, difficulty={1}, origin={bash}]
  O programa \lstinline+cp+ copia um arquivo.

  O programa \lstinline+mv+ move um arquivo.

  \Question Escreva o comando que copia o arquivo \lstinline+foo+ para o
  arquivo \lstinline+bar+.
  \Question Escreva o comando que move o arquivo \lstinline+foo+ para o
  arquivo \lstinline+bar+.
\end{Exercise}
\begin{Answer}[ref={0018}]
  \Question \lstinline+cp foo bar+
  \Question \lstinline+mv foo bar+
\end{Answer}

\begin{Exercise}[label={0019}, difficulty={2}, origin={bash}]
  Escreva o comando que copia a pasta \lstinline+foo+ para a
  pasta \lstinline+bar+ (é necessário copiar o conteúdo).
\end{Exercise}
\begin{Answer}[ref={0019}]
  \Question \lstinline+cp -r foo bar+
\end{Answer}

\begin{Exercise}[label={0020}, difficulty={1}, origin={bash}]
  O programa \lstinline+mkdir+ cria um novo diretório.

  \Question Escreva o comando para criar o diretório \lstinline+foo+.
  \Question Escreva o comando para criar os diretórios \lstinline+foo+ e
  \lstinline+bar+.
\end{Exercise}
\begin{Answer}[ref={0020}]
  \Question \lstinline+mkdir foo+
  \Question \lstinline+mkdir foo bar+
\end{Answer}

\begin{Exercise}[label={0021}, difficulty={2}, origin={bash}]
  Escreva o comando para criar o diretório \lstinline+foo/bar/aux+ supondo que
  os diretórios \lstinline+foo+ e \lstinline+foo/bar+ não existem.
\end{Exercise}
\begin{Answer}[ref={0021}]
  \lstinline+mkdir -p foo/bar/aux+
\end{Answer}

\begin{Exercise}[label={0022}, difficulty={1}, origin={bash}]
  O programa \lstinline+rm+ remove arquivos (e diretórios).

  \Question Escreva o comando para remover o arquivo \lstinline+foo+.
  \Question Escreva o comando para remover os arquivos \lstinline+foo+ e
  \lstinline+bar+.
  \Question Escreva o comando para remover o diretório \lstinline+foo+.
\end{Exercise}
\begin{Answer}[ref={0022}]
  \lstinline+rm foo+
  \lstinline+rm foo bar+
  \lstinline+rm -rf foo+
\end{Answer}
