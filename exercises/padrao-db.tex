\begin{Exercise}[label={easy}, difficulty=0]
  Esta é uma questão fácil, \verb+difficulty=0+.
\end{Exercise}
\begin{Answer}[ref={easy}]
  Resposta de um exercício fácil.
\end{Answer}
\begin{Exercise}[label={medium}, difficulty=1]
  Esta é uma questão média, \verb+difficulty=1+.
\end{Exercise}
\begin{Answer}[ref={medium}]
  Resposta de um exercício médio.
\end{Answer}
\begin{Exercise}[label={hard}, difficulty=2]
  Esta é uma questão difícil, \verb+difficulty=2+.
\end{Exercise}
\begin{Answer}[ref={hard}]
  Resposta de um exercício difícil.
\end{Answer}
\begin{Exercise}[label={veryhard}, difficulty=3]
  Esta é uma questão muito difícil, \verb+difficulty=3+.
\end{Exercise}
\begin{Answer}[ref={veryhard}]
  Resposta de um exercício muito difícil.
\end{Answer}
\begin{Exercise}[label={lstlisting}, difficulty=0]
  Esta é uma questão que mostra o uso do pacote \verb+lstlisting+.
  \lstinputlisting{aux/padrao.c}
  Códigos e arquivos que o aluno passa precisar devem ser
  ``impressos''/incluidos no exercício utilizando o comando
  \verb+lstinputlisting+ do pacote \verb+lstlisting+ pois o \verb+Makefile+ gera
  automaticamente um arquivo \verb+tar+ contendo estes arquivos.
\end{Exercise}
\begin{Answer}[ref={lstlisting}]
  Resposta de um exercício.
\end{Answer}
