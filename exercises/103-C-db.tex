\begin{Exercise}[label={0001}, difficulty={0}, origin={io}]
  Qual é a função que imprimi uma string na tela e a qual biblioteca ela
  pertence?
\end{Exercise}
\begin{Answer}[ref={0001}]
  A função se chama \lstinline+printf+ e pertence a biblioteca
  \lstinline+stdio.h+.
\end{Answer}

\begin{Exercise}[label={0002}, difficulty={0}, origin={io}]
  Faça um programa que lê dois números reais e imprime seu produto.
\end{Exercise}
\begin{Answer}[ref={0002}]
\begin{lstlisting}
#include <stdio.h>

int main () {
  float x, y;

  scanf("%f %f", &x, &y);
  printf("%f * %f = %f\n", x, y, x*y);
  
  return 0;
}
\end{lstlisting}
\end{Answer}

\begin{Exercise}[label={0003}, difficulty={1}, origin={io}]
  Considere o código
\begin{lstlisting}
#include <stdio.h>

int main () {
  float x = 3.1416;

//  printf
  return 0;
}
\end{lstlisting}
e a saída
\begin{lstlisting}[showspaces=true]
   3.142
\end{lstlisting}
  Como deve ser a chamada do printf para obter essa saída? Preste atenção na quantidade
  de espaços na saída.
\end{Exercise}
\begin{Answer}[ref={0003}]
\begin{lstlisting}[firstnumber=6]
printf("%8.3f\n", x);
\end{lstlisting}
\end{Answer}

\begin{Exercise}[label={0004}, difficulty={1}, origin={type}]
  Qual o erro do programa abaixo?
\begin{lstlisting}
#include <stdio.h>

int main () {
  int x = 0;

  printf("%f\n", x+1);
  return 0;
}
\end{lstlisting}
\end{Exercise}
\begin{Answer}[ref={0004}]
  O printf está errado. Deveria ser \lstinline+%d+ no lugar de
  \lstinline+%f+.
\end{Answer}

\begin{Exercise}[label={0005}, difficulty={1}, origin={if}]
  Faça um programa que lê um número real e imprime o módulo
  desse número.
\end{Exercise}

\begin{Exercise}[label={0006}, difficulty={1}, origin={if}]
  Faça um programa que lê três números reais e os imprime em
  ordem crescente.
\end{Exercise}

\begin{Exercise}[label={0007}, difficulty={1}, origin={for}]
  Faça um programa que lê um número real $x$ e um inteiro $n$ e imprime
  $x^n$.
\end{Exercise}

\begin{Exercise}[label={0008}, difficulty={1}, origin={for}]
  Faça um programa que lê um número inteiro $n$ e imprime um triângulo
  no formato
\begin{lstlisting}
*
**
***
****
\end{lstlisting}
  com $n$ linhas.
\end{Exercise}

\begin{Exercise}[label={0009}, difficulty={2}, origin={for}]
  Faça um programa que lê um número inteiro $n$ e
  \Question se $n$ for 0 ou negativo, retorna uma mensagem de erro;
  \Question se $n$ é positivo, imprime todos os divisores de $n$.
\end{Exercise}

\begin{Exercise}[label={0010}, difficulty={3}, origin={for}]
  Faça um programa que lê um número real $x$ e  calcula $e^x$ 
  utilizando a aproximação
  $$ e^x = 1 + x + \frac{x^2}{2} + \frac{x^3}{3!} + \frac{x^4}{4!} +
  \dots + \frac{x^N}{N!} = \sum_{k=0}^N\frac{x^k}{k!} $$
  usando $N = 50$. Sua implementação deve
  \Question verificar se $x$ é nulo e, caso positivo, retornar o
    valor esperado trivialmente;
  \Question conseguir calcular $e^{30}$. Para isso, é preciso
    tomar muito cuidado com a maneira de calcular os termos.
\end{Exercise}
